%--------------------
% Packages
% -------------------
\documentclass[11pt,a4paper]{article}


\usepackage[pdftex]{graphicx} % Required for including pictures
\usepackage[pdftex,linkcolor=black,pdfborder={0 0 0}]{hyperref} % Format links for pdf
\usepackage{calc} % To reset the counter in the document after title page

\frenchspacing % No double spacing between sentences
\linespread{1.2} % Set linespace
\usepackage[a4paper, lmargin=0.1666\paperwidth, rmargin=0.1666\paperwidth, tmargin=0.1111\paperheight, bmargin=0.1111\paperheight]{geometry} %margins

\usepackage[protrusion=true,expansion=true]{microtype} % Improves typography, load after fontpackage is selected


%-----------------------
% Set pdf information and add title, fill in the fields
%-----------------------
\hypersetup{
pdfsubject = {Software Product Engineering},
pdftitle = {SCEEM Space - Product Evaluation},
pdfauthor = {Jason Park, Sungijn Kang, William Nafack, Calum West}
}

%-----------------------
% Begin document
%-----------------------
\begin{document}

\begin{titlepage}
   \vspace*{\stretch{1.0}}
   \begin{center}
      \Large\textbf{SCEEM Space - Product Evaluation}\\
      \large\textit{Jason Park, Sungijn Kang, William Nafack, Calum West}
   \end{center}
   \vspace*{\stretch{2.0}}
\end{titlepage}

\section{Product Evaluation}
Our approach to evaluating our product towards the end of development was mostly through the meetings we held with our client. We used these to show progress we had made and gave the client an opportunity to outline anymore features they wanted adding to the end product.
\bigskip

During these meetings, the main form of evaluation was through interview style questions and talking through the views for the product with the client. When any of the group had a question to ask the client, we asked it, and recorded the answers, making extra notes where suitable. This approach worked well because we were then able to go over our notes after the meeting and collate them into specific tasks that needed to be completed. We then added these tasks to Jira and assigned them to different members of the group.
\bigskip

We also evaluated the product through observation, this was done in the meetings where we allowed the client to use the product we had so far. Watching how they used the system gave us a good idea of the flow steps for a typical use case and allowed us to make small changes to make the product more user friendly.
\bigskip

Both of these approaches worked well and gave us the most information. Meeting face to face with the client for product evaluation was much better than communicating via emails as it made it easier to get different points across and convey information that would've been much harder to do so via email.
\bigskip

It was very hard to evaluate the final product we produced through end user testing. This was due to the COVID-19 pandemic meaning we were unable to contact our clients as emails were not being responded to. If this had not been the case we would have created questionnaires to provide insight into how the final product could be improved. Ones with short questions, giving feedback and what was already good and what needed improving would have given us adequate feedback so that we could construct a list of tasks that needed doing to improve the system.
\bigskip

The client themselves were going to be the final users of the system so it made sense to have them giving feedback throughout development. There are no other users to have evaluate the final product. However, since we were not able to contact our client, we had to perform the final evaluation of our system ourselves. We did this through testing (although our system had been tested throughout development aswell) and also by using the system in the same way we thought the clients would be using the product.

\end{document}